\documentclass[12pt]{report}

\usepackage[french]{babel}

\usepackage[a4paper,top=2cm,bottom=2cm,left=2cm,right=2cm]{geometry}

\usepackage[T1]{fontenc}
\usepackage{amsmath}
\usepackage{graphicx}
\usepackage[colorlinks=true, allcolors=black]{hyperref}
\usepackage{titlesec}
\usepackage{lipsum}

\titleformat{\chapter}[display]
{\normalfont\bfseries}{}{0pt}{\huge}


\title{{\Huge Rapport de Projet} \\Asservissement du turboréacteur DGEN 380 }
\author{RICHELET Arthur
\\RENAYUD Maxime
\\FONTANELLE Dorian
\and
  DELMOTTE-DIAS Hélène
\\PASQUET Xavier}


\date{Encadrant: M. \\ \vspace{0.5cm}
Décembre 2022 \\
\vspace{2cm}
\includegraphics[scale=0.4]{fig/evering_logo.jpg}}

\begin{document}
\maketitle

\tableofcontents


\newpage

\chapter{Abstract}

Le but de ce projet est de réaliser un asservissement
du turboréacteur DGEN 380.
Dans ce petit moteur de jet, la vitesse de rotation de la turbine 
est liée à la quantité de carburant injecté dans la chambre de 
combustion mais aussi des perturbations externes comme la température,
la pression de l'air ambiant ou la vitesse de l'air entrant dans le compresseur.\newline
 
L'objectif est de déterminer une loi de commande permettant d'atteindre un certain
régime moteur en fonction de la position de la manette des gaz. L'outil principal
utilisé pour la réalisation de ce projet est le logiciel {\it MATLAB/Simulink}.


%% --
\chapter{Introduction}
\section{DGEN 380} %section d'introduction au DGEN 380
\lipsum[1-2]

\section{}

\chapter{Méthodologie}


\end{document}